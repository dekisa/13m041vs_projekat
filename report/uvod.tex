\section{Uvod}
U ovom radu je na DE1-SoC razvojnom sistemu implementiran jednostavan hardver u FPGA, portovan je Linuks operativni sistem i napisan je drajver za pristup registrima i prihvatanje prekida iz FPGA.

\subsection{Sistemi na čipu sa FPGA}
Sa sve većim mogućnostima namenskih sistema došlo je do popularizacije sistema na čipovima (SoC - \textit{System on Chip}) koji integrišu mikroprocesore sa više jezgara, memorije na čipu, mnogobrojne periferije i transivere, kao i FPGA (\textit{Field Programmable Gate Array}).

\subsection{Opis DE1-SoC}


U nastavku su navedene samo osobine razvojnog sistema koje se tiču ovog rada, a detaljniji opis se moze pronaći u dokumentu zvaničnoj dokumentaciji proizvođača \cite{de1}: 
\begin{itemize}
\item Sistem na čipu Cyclone V \texttt{5CSEMA5F31}
\item Memorija 1GB (2x256Mx16) DDR3 SDRAM povezana na HPS
\item Slot za Micro SD karticu povezan na HPS
\item UART na USB (USB Mini-B konektor)
\item 5 debaunsiranih tastera (FPGA x4, HPS x1)
\item 11 LE dioda (FPGA x10, HPS x 1)
\item 12V DC napajanje
\end{itemize}

\subsection{Opis Altera Cyclone V}
Altera Cyclone V je SoC FPGA koji se sastoji od dva dela: procesorski deo (HPS -  \textit{Hard processor System}) i programabilni FGPA deo. HPS se sastoji od MPU (\textit{Microprocessor unit}) sa ARM Cortex-A9 MPCore sa dva jezgra i sledećih modula: kontroleri memorije, memorije, periferije, sistem interkonekcije, debug moduli, PLL moduli. FPGA deo se sastoji od sledećih delova: FPGA programabilna logika (\textit{look-up} tabele, RAM memorije, mnozači i rutiranje), kontrolni blok, PLL, kontroler memorije.

\subsubsection{Konfigurisanje FPGA i pokretanje HPS}
Pri pokretanju HPS (\textit{boot}) može da učita program iz FPGA dela, iz eksterne \textit{flash} memorije ili preko JTAG. FPGA ima mogućnost da se konfiguriše softverski iz HPS korišćenjem periferije \textit{FPGA Manager} ili spoljnim programatorom. Kombinacije ovih mogućnosti daju nekoliko scenarija:

Pokretanje HPS je proces koji se obavlja u više koraka. Nakon izvršavanja svakog koraka se učitava i pokreće sledeći. Ovo je proces je sličan kod svih ARM procesora, a u nastavku je ukratko opisan za konkretnu platformu. Grafički prikaz procesa je na slici \ref{slika1:bootr}

\begin{figure}[h!]
\centering
\includegraphics[scale=1.]{img/gsrd-boot.png}
\caption{Tok pokretana sistema}
\label{slika1:bootr}
\end{figure}

U nastavku su objašnjeni fajlovi koji se koriste pri projektovanju:
\begin{itemize}
\item \texttt{.qpf} - projektni fajl za Quartus. Ovaj fajl generiše DE1-SoC Builder
\item \texttt{.qsf} - skripta za podešavanje pinova. Ovaj fajl generiše DE1-SoC Builder
\end{itemize} 
\pagebreak
\begin{lstlisting}[language=C]
/ {
	#address-cells = <1>;
	#size-cells = <1>;
	
	cpus {
		#address-cells = <1>;
		#size-cells = <0>;

		cpu0: cpu@0 {
			compatible = "arm,cortex-a9";
			device_type = "cpu";
			reg = <0>;
		};
		cpu1: cpu@1 {
			compatible = "arm,cortex-a9";
			device_type = "cpu";
			reg = <1>;
		};
	};
\end{lstlisting}